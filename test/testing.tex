\documentclass[margin=1cm,varwidth]{standalone}
\usepackage{graphicx}
\usepackage{float}
\usepackage{caption}
\usepackage{subcaption}
\usepackage{pgffor}
\usepackage{calc}
\usepackage{tikz}
\usepackage{xstring}
\usepackage{listofitems}
\usepackage{fp}
\usepackage{pgf}
\usepackage{xfp}
\usepackage{../feynman-colorflow}


\newcommand{\addIfNegative}[1]{%
    \ifnum#1<0
        \the\numexpr#1+360\relax
    \else
        #1
    \fi
}

\newcommand{\smallest}[2]{%
    \ifnum#1<#2
        #1
    \else
        #2
    \fi
}

\newcommand{\biggest}[2]{%
    \ifnum#1<#2
        #2
    \else
        #1
    \fi
}

% \newcommand{\addIfNegative}[1]{\fpeval{#1 < 0 ? #1 + 360 : #1}}
% \newcommand{\biggest}[2]{\fpeval{#1 > #2 ? #1 : #2}}
% \newcommand{\smallest}[2]{\fpeval{#1 < #2 ? #1 : #2}}

\newcommand{\compareAngles}[2]{%
  \edef\angleA{\addIfNegative{#1}}
  \edef\angleB{\addIfNegative{#2}}
  
  \ifnum\fpeval{abs(\angleA - \angleB) > 180}=1
    \fpeval{\biggest{\angleA}{\angleB}}
  \else
    \fpeval{\smallest{\angleA}{\angleB}}
  \fi
}

% \newcommand{\compareNumbers}[2]{%
%     \ifnum\fpeval{abs(\addIfNegative{#1} - \addIfNegative{#2}) > 180}=1
%       \fpeval{(360 - abs(\addIfNegative{#1} - \addIfNegative{#2}))/2}
%     \else
%       \fpeval{abs(\addIfNegative{#1} - \addIfNegative{#2})/2}
%     \fi
% }

% \newcommand{Smallest}[2]
% {
%   \ifthenelse{abs(\addIfNegative{#1} - \addIfNegative{#2}) < 180}
%     \addIfNegative{\smallest{#1}{#2}}
%   \ifthenelse{abs(\addIfNegative{#1} - \addIfNegative{#2}) > 180}
%     \addIfNegative{\biggest{#1}{#2}}
%   }

\newcommand{\Smallest}[2]{%
  \ifthenelse{abs(\addIfNegative{#1} - \addIfNegative{#2}) < 180}{%
    \addIfNegative{\smallest{#1}{#2}}%
  }{%
    \ifthenelse{abs(\addIfNegative{#1} - \addIfNegative{#2}) > 180}{%
      \addIfNegative{\biggest{#1}{#2}}%
    }{%
      % Handle the case when the absolute difference is exactly 180.
      % You can add code for this situation if needed.
    }%
  }%
}  

% \newcommand{\TTest}[2]
% {
%   \ifthenelse{#1>#2}{#2}
%   \ifthenelse{#1<#2}{#1}
% }

\newcommand{\TTestA}[2]{%
  \ifthenelse{abs(\addIfNegative{#1} - \addIfNegative{#2}) < 180}{%
    #2% Return #2 if #1 is greater
  }{%
    \ifthenelse{#1 < #2}{%
      #1% Return #1 if #2 is greater
    }{%
      % Handle the case when #1 and #2 are equal (you can add code or logic here)
    }%
  }%
}
\newcommand{\convertNegativeAngle}[1]{%
  \ifnum#1<0
    \the\numexpr#1+360\relax
  \else
    #1
  \fi
}


\newcommand{\TTest}[2]{%
  \def\angleA{\convertNegativeAngle{#1}}
  \def\angleB{\convertNegativeAngle{#2}}
  
  \ifthenelse{abs(\angleA - \angleB) < 180}{%
    \angleB% Return angleB if angleA is greater
  }{%
    \ifthenelse{\angleA < \angleB}{%
      \angleA% Return angleA if angleB is greater
    }{%
      % Handle the case when angleA and angleB are equal or the absolute difference is exactly 180.
      % You can add code or logic here.
    }%
  }%
}

% Define a macro to convert negative angles to positive angles




\newcommand{\compareNumbers}[2]{%
  \pgfmathtruncatemacro{\absdiff}{abs(\addIfNegative{#1} - \addIfNegative{#2})}
  \pgfmathsetmacro{\result}{%
    ifthenelse(\absdiff > 180, (360 - \absdiff)/2, \absdiff/2)
  }
  \result
}
\newcommand{\SharedCoordinate}[5]{
  \pgfmathsetmacro{\angle}{abs(\addIfNegative{#3} - \addIfNegative{#4})/2}
  \pgfmathsetmacro{\radius}{#5/sin(\angle)}
  \pgfmathsetmacro{\xcoord}{\radius*cos(\smallest{\addIfNegative{#3}}{\addIfNegative{#4}}+\angle)}
  \pgfmathsetmacro{\ycoord}{\radius*sin(\smallest{\addIfNegative{#3}}{\addIfNegative{#4}}+\angle)}  

  \coordinate (#1) at ($ (#2) + (\xcoord, \ycoord) $);
}

\newcommand{\SharedCoordinateD}[5]{
  \pgfmathsetmacro{\angle}{\compareNumbers{#1}{#2}}
  \pgfmathsetmacro{\radius}{#5/sin(\angle)}
  \pgfmathsetmacro{\xcoord}{\radius*cos(\biggest{\addIfNegative{#3}}{\addIfNegative{#4}}+\angle)}
  \pgfmathsetmacro{\ycoord}{\radius*sin(\biggest{\addIfNegative{#3}}{\addIfNegative{#4}}+\angle)}  

  \coordinate (#1) at ($ (#2) + (\xcoord, \ycoord) $);
}

\newcommand{\SharedCoordinateB}[5]{
  \pgfmathsetmacro{\angle}{\fepval\compareNumbers{#3}{#4}}
  \pgfmathsetmacro{\radius}{#5/sin(\angle)}
  \pgfmathsetmacro{\xcoord}{\radius*cos(\compareAngles{#3}{#4}+\angle)}
  \pgfmathsetmacro{\ycoord}{\radius*sin(\compareAngles{#3}{#4}+\angle)}  

  \coordinate (#1) at ($ (#2) + (\xcoord, \ycoord) $);
}

% \newcommand{\SharedCoordinateA}[5]{
%   \compareNumbers{#3}{#4}
%   \pgfmathsetmacro{\angle}{\result}
%   \compareAngles{#3}{#4}
%   \pgfmathsetmacro{\radius}{#5/sin(\angle)}
%   \pgfmathsetmacro{\xcoord}{\radius*cos(\result+\angle)}
%   \pgfmathsetmacro{\ycoord}{\radius*sin(\result+\angle)}  

%   \coordinate (#1) at ($ (#2) + (\xcoord, \ycoord) $);
% }
% \AtBeginDocument{%
%   \DTLgnewdb{DV1}%
% }
\newcommand{\SharedCoordinateA}[5]{
  % Calculate the angle based on the difference between #3 and #4
  \pgfmathsetmacro{\angle}{abs(#3 - #4) > 180 ? 360 - abs(#3 - #4) : abs(#3 - #4)}

  % Calculate the radius
  \pgfmathsetmacro{\radius}{#5 / sin(\angle)}

  % Calculate the x and y coordinates
  \pgfmathsetmacro{\xcoord}{\radius * cos(\angle + #3)}
  \pgfmathsetmacro{\ycoord}{\radius * sin(\angle + #3)}

  % Define the coordinate
  \coordinate (#1) at ($ (#2) + (\xcoord, \ycoord) $);
}

\begin{document}

% \compareNumbers{0}{200}
% \compareAngles{0}{200}
% \compareAngles{-10}{90}
% \compareNumbers{0}{90}
% \tikzmath{\r1=0.05; \rr=0.01; }
% \TTest{10}{20}
% \Smallest{10}{30}
% \begin{tikzpicture}
%   \firstvertex{V1}{f,B,F}{-180,-90,0}
%   % \Labels{V1}{q}{left}{1.15}
%   \vertex{V2}{V1}{2}{f,F}{-180,0}
%   \ParticleLabel{V1}{90}{midway=0.4}{1.15}{test}
%   % \SharedCoordinate{S11}{V1}{-180}{-90}{0.3}
%   % \draw[fill=red,draw=red] (S11) circle [radius=\r1];
%   % \SharedCoordinate{S11}{V2}{-180}{90}{0.3}
%   % \draw[fill=red,draw=red] (S11) circle [radius=\r1];
% \end{tikzpicture}



% \begin{tikzpicture}
%   \Labels{V1}{q}{left}{1.15}
% \end{tikzpicture}

\begin{tikzpicture}
  \firstvertex{V1}{f,f,F}{0,110,200}
  \vertex{V2}{V1}{1}{G,G}{45,-45}
  \vertex{V3}{V2}{1}{G,G}{45,-30}

  \SharedCoordinate{S1}{V1}{200}{110}{0.4}
  \draw[fill=red,draw=red] (S1) circle [radius=\r1];
  
  \SharedCoordinate{S11}{V1}{0}{200}{0.4}
  \draw[fill=red,draw=red] (S11) circle [radius=\r1];

  \SharedCoordinateA{S122}{V1}{0}{200}{0.4}
  \draw[fill=blue,draw=red] (S122) circle [radius=\r1];
  
  % \SharedCoordinate{S2}{V2}{-180}{-45}{0.4}
  % \draw[fill=red,draw=red] (S2) circle [radius=\r1];
  
  % \SharedCoordinate{S3}{V3}{-110}{135}{0.4}
  % \draw[fill=red,draw=red] (S3) circle [radius=\r1];

  \coordinate (S11) at ($ (V1) + ({2*cos(110+atan(0.4/2))},{2*sin(110+atan(0.4/2))}) $);
  \draw[fill=red,draw=red] (S11) circle [radius=\r1];

  % \coordinate (S12) at ($ (V1) + 0.2*({2*cos(270},{2*sin(270}) $);
  % \draw[fill=red,draw=red] (S12) circle [radius=\r1];

\end{tikzpicture}

\begin{tikzpicture}
  \firstvertex{V1}{g,g,G}{-135,135,0}
  \vertex{V2}{V1}{3}{F,f}{-45,45}
  \vertex{V4}{V2}{1}{F,B}{-25,25}
  \vertex{V3}{V2}{2}{B,f}{-25,25}
\end{tikzpicture}

% \readlist\ParticleName{f1,F1,g}

% \ParticleName[2]
% \readlist\Angle{100,200,0}
% \Angle[1]
% \DTLnewdb{DV1}
  
% \DTLnewrow{DV1}
% \DTLnewdbentry{DV1}{key}{\ParticleName[1]}
% \DTLnewdbentry{DV1}{value}{\Angle[1]}

% \DTLnewrow{DV1}
% \DTLnewdbentry{DV1}{key}{\ParticleName[2]}
% \DTLnewdbentry{DV1}{value}{\Angle[2]}

% \DTLnewrow{DV1}
% \DTLnewdbentry{DV1}{key}{\ParticleName[3]}
% \DTLnewdbentry{DV1}{value}{\Angle[3]}

% \DTLnewrow{DV1}
% \DTLnewdbentry{DV1}{key}{\ParticleName[4]}
% \DTLnewdbentry{DV1}{value}{\Angle[4]}



% \searchDictionary{particles}{f}
% \begin{tikzpicture}
%   \coordinate (O) at (0,0);
%   \VertexStart{V1}{f,F,G}{-180,-90,0}{f1,F1,G1}
%   \VertexCont{V2}{V1}{0}{G,G}{-45,45}
%   \VertexCont{V3}{V1}{-90}{f,B}{-90,0}
%   \VertexCont{V4}{V3}{-90}{f,G}{-180,0}
%   \draw[color=red, line width = \s, decoration={markings, mark=at position 0.5 with {\arrow{>}}}, postaction={decorate}] (V1)--(V2);
% \end{tikzpicture}

\end{document}

% \newcommand{\compareAngles}[2]
% {
%   \ifnum\fpeval{abs(\addIfNegative{#1} - \addIfNegative{#2}) > 180}=1
%     \fpeval{#1}
%     \fpeval{\biggest{\addIfNegative{#1}}{#2}}
%   \else
%     \fpeval{\smallest{#1}{#2}}
%   \fi
% }
